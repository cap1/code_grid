\section{Aufgabenblatt 3}

\subsection{Torque Portable Batch System}
\subsubsection{Beschreibe die PBS Befehle}

\begin{description}
	\item[\texttt{qsub}] \hfill \\
		Mit diesem Befehl lassen sich neue Jobs an in eine Warteschlange eintragen.
		Durch verschiedene Schalter lassen sich die Parameter \"ubergeben,
		beispielsweise für eine mindestens benötigte Menge an Arbeitsspeicher.

	\item[\texttt{qdel}] \hfill \\
	\item[\texttt{qstat}] \hfill \\
	\item[\texttt{pbsnodes}] \hfill \\
\end{description}


\subsection{PBS System im Informatik Pool}
\subsubsection{Wieviele Knoten sind vorhanden?}
\subsubsection{Wie lassen sich alle freien Knoten anzeigen?}
\subsubsection{Welche Warteschlangen sind definiert und wie lange d\"urfen Jobs laufen?}


\subsection{Paralellisierung mit PBS}

\subsubsection{POV-Ray}
