\section{Aufgabenblatt 3}

\subsection{Torque Portable Batch System}
\subsubsection{Beschreibe die PBS Befehle}

\begin{description}
	\item[\texttt{qsub}] \hfill \\
		Mit diesem Befehl lassen sich neue Jobs an in eine Warteschlange eintragen.
		Durch verschiedene Schalter lassen sich die Parameter \"ubergeben,
		beispielsweise für eine mindestens benötigte Menge an Arbeitsspeicher.

		Üblicherweise wir \texttt{qsub} ein Shell-Script übergeben,
		welches den Aufruf übernimmt und die Parameter enthält.
		Zusätzlich können in diesem Script noch Information an PBS übergeben werden.
		Dies sind zum einen Konfigurationsparamenter für PBS,
		zum anderen Kontaktinformationen zum Besitzer des Jobs.

	\item[\texttt{qdel}] \hfill \\
		Mit diesem Befehl lassen sich in der Warteschlange stehende Jobs aus dieser entfernen.
		Dazu muss dem Befehl die von \texttt{qsub} zurückgelieferte Job-ID übergeben werden.

	\item[\texttt{qstat}] \hfill \\
		Diser Befehl erlaub das Verfolgen eines dem Grid übergebenen Jobs mittels der Job-ID.

	\item[\texttt{pbsnodes}] \hfill \\
\end{description}


\subsection{PBS System im Informatik Pool}
\subsubsection{Wieviele Knoten sind vorhanden?}
\subsubsection{Wie lassen sich alle freien Knoten anzeigen?}
\subsubsection{Welche Warteschlangen sind definiert und wie lange d\"urfen Jobs laufen?}


\subsection{Paralellisierung mit PBS}

\subsubsection{POV-Ray}
