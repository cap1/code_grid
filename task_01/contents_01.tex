\section{Aufgabenblatt 1}

\subsection{Aufgabe 1 - Grid Computing}

\subsubsection{Definiere Grid}
	\quotation{A computational grid is a hardware and software infrastructure that provides dependable, consistent, pervasive, and inexpensive access to high-end computational capabilities.}
	In einem Grid werden Ressourcen verwaltet und verteilt.
	Zus\"atzliche Resourcen k\"onnen bei Bedarf im Grid bereitgestellt werden.
	Die Ressourcen k\"onnen dabei auf verschiedene Weise gekoppelt werden:
	Sequentiell, verteilt oder parallel.
	Ein Grid wird von mehreren Personen oder Gruppen (VO) gleichzeitig verwendet.
	Sie \"ubergeben ihre Aufgaben dem Grid, welches dann die bestm\"ogliche Art und Weise der Ausf\"uhrung ermittelt.
	Die Infrastruktur eines Grids besteht aus einzelnen Knoten,
	die die eingentlichen Auftr\"age abarbeiten.
	Sie werden durch das Grid jedoch vor dem Nutzer versteckt und die Zuordnung der
	Aufgaben erfolgt durch spezielle Software (Grid Middleware).
	
\subsubsection{Vorteile von Grid-Technologie}
	\begin{itemize}
	  \item Gemeinsames Nutzen von teurer Hardware
	  \item Vereinfachung der Nutzung und erh\"ohte Portabilit\"at
	\end{itemize}
	
\subsubsection{Anwendungen im Grid und deren Besonderheiten}
	Anwendungen im Grid m\"ussen in der Lage sein,
	die ihnen gestellten Aufgaben in einzelne zu zerteilen.
	Diese Aufgaben k\"onnen dann von den einzelnen Knoten verarbeitet werden.
	Dazu sind unter Umst\"anden bestimmte Anpassungen n\"otig.
	
\subsubsection{Definiere Grid-Middleware}
	Ein Grid-Middleware ist daf\"ur verantwortlich,
	dass sich ein Grid verh\"alt wie ein einzelnes System.
	
	Sie vermittelt zwischen den laufenden Anwenungen und den lokalen
	Betriebssystemen der einzelnen Rechner/Knoten.

	%Nicht auch accounting und blubbs?
	
\subsection{Cluster Computing}
	\subsubsection{Skizziere eine Cluster Architektur und beschreibe die
	Wichtigsten Komponenten}
	\subsubsection{Welche Warteschlangenstrategien gibt es? Welche w\"urdest du
	bevorzugen? Begr\"unde!}
	
\subsection{Parallelisierung}
