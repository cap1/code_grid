\section{Aufgabenblatt 1}

\subsection{Aufgabe 1 - Grid Computing}

\subsubsection{Definiere Grid}
	\quotation{A computational grid is a hardware and software infrastructure that provides dependable, consistent, pervasive, and inexpensive access to high-end computational capabilities.}
	In einem Grid werden Ressourcen verwaltet und verteilt.
	Zusaetzliche Resourcen koennen bei Bedarf im Grid bereitgestellt werden.
	Die Ressourcen koenn dabei auf Verschiedene Weise gekoppelt werden:b
	Sequentiell, Verteilt oder Parallel
	Ein Grid wird von mehreren Personen oder Gruppen (VO) gleichzeitig verwendet.
	Sie geben ihre uebergeben ihre Aufgaben dem Grid,
	welches dann die bestmoegliche Art und Weise der Ausfuehrung ermittelt.
	
	Die Infrastruktur eines Grids besteht aus einzelnen Knoten,
	die die eingelntlichen Auftraege abarbeiten.
	Sie werden durch das Grid jedoch vor dem Nutzer versteckt und die Zuordnung der
	Aufgaben erfolgt durch spezielle Software (Grid Middleware).
	
\subsubsection{Vorteile von Grid-Technologie}
	\begin{itemize}
	  \item Gemeinsames Nutzen von teurer Hardware
	  \item Vereinfachung der Nutzung und erhoehte Portabilitaet
	\end{itemize}
	
\subsubsection{Anwendungen im Grid und deren Besonderheiten}
	Anwendungen im Grid muessen in der Lage sein,
	die ihnen gestellten Aufgaben in einzelne zu zerteilen.
	Diese Aufgaben koennen dann von den einzelnen Knoten verarbeitet werden.
	Dazu sind unter umstaenden bestimmte Anpassungen noetig.
	
\subsubsection{Definiere Grid-Middleware}
	Ein Grid-Middleware ist dafuer verantwortlich,
	dass sich ein Grid verhaelt wie ein einzelnes System.
	
	Sie vermittelt zwischen den laufenden Anwenungen und den lokalen
	Betriebssystemen der einzelnen Rechner/Knoten.
	
\subsection{Cluster Computing}
	\subsubsection{Skizziere eine Cluster Architektur und beschreibe die
	Wichtigsten Komponenten}
	\subsubsection{Welche Warteschlangenstrategien gibt es? Welche wuerdest du
	bevorzugen? Begruende!}
	
\subsection{Parallelisierung}