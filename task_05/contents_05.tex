\section{Aufgabenblatt 5}
\subsection{Cluster und Grid Computing}

	\subsubsection{Zusammenhänge von Grid und Cluster Systemen}
	{\scriptsize Stichpunkte: \textsl{Virtual Organization, Grid Site, Cluster Management, Cluster Clients, Grid Server, Grid Clients, Information Service, Meta-Scheduler} } \\
	
		Grid Systeme sind lose Zusammenschlüsse von Computersystemen.
		Sie sind meist noch heterogener zusammengesetzt als Cluster Systeme.
		Grid werden von Virtuellen Organisationen verwendet.
		Sie gruppieren Gruppen von Benutzer nach definierten Eingenschaften
		und legen der Zugriffsrechte innerhalb des Grids fest.
		Die Verwaltung von Grid-Systemen erfolgt über eine Grid Middleware,
		welche der verbindende Punkt zwischen den einzelnen Grid Sites ist.
		
		Cluster können teile von Grids sein.
		Die Cluster werden dann zu Grid-Clients,
		welche durch den Meta-Scheduler des Grids Jobs zugewiesen bekommen.
		Die Jobs werden dann im Cluster auf die einzelnen Nodes verteilt.
		
		Das Grid-System wir durch den Information Service zusammengehalten.
		Es ermöglicht den Zugriff auf wichtige Daten des Gridsystems.
		Beispielsweise, welche Daten in welcher Gridsite lokal vorhanden sind,
		oder wie stark eine Ressource ausgelastet ist, bzw. verfügbar ist.

	\subsubsection{Was ist ein Meta-Scheduler?}
		\label{sec:metascheduler}
		Ein Meta-Scheduler vereint alle Scheduler eines Grids und
		verteilt die Jobs an den jeweils optimalen Standort.
		%ws-gram
		
	\subsubsection{Wozu benötigt man den Information Service?}
		Er stellt Informationen über die Bestandteile eines Grids zur Verfügung.
		Durch die geographische Verteilung der Grid-Ressourcen ist es nötig,
		automatisch maschinenlesbare Informationen über die Grid-Ressourcen zu erhalten.
		Insbesondere der Meta-Scheduler (\ref{sec:metascheduler}) ist darauf angewiesen.

		Im Globus Toolkit übernimmt der \textsl{Monitoring and Discovery Service} diese 
		Aufgabe.
		
	\subsubsection{Vorteile von Grid gegenüber Clustersystemen}
		Grid Systeme sind meist größer als einzelne Cluster.
		Abhängig von der Größe des Jobs und des Entstehenden Overheads
		durch Verwendung des Grids,
		muss abgewogen bzw. abgeschätzt werden wo der Job schneller beendet wird.
		Bei der Verwendung des Grids muss in Betracht gezogen werden,
		dass möglicher Weise große Datenmengen transportiert werden müssen
		um Jobs zu starten.

\subsection{Zertifikate und Proxy}
	\subsubsection{Wofür werden Zertifikate im Grid eingesetzt?\\
	Wie sieht der Prozess von Beantragung \& Ausstellung aus?}
		Zur Identifikation von Entitäten (Hosts, Dienste, Nutzer, usw.) 
		und für verschlüsselte Kommunikation im Grid.
		Die X.509 Zertifikate werden von Zertifizierungsstellen (CA) signiert.
		Besitzer eines signierten Zertifikates können kurzlebige Proxy-Zertifikate erstellen,
		um ihre Rechte weiterzugeben.
		
	\subsubsection{Zusammenhang: Grid-Zertifikat, -Credentials, -Proxy, Credentials Delegation und öffentlicher Schlüssel}
		Ein Grid-Zertifikat besteht aus einem öffentlichen und einem privaten Teil.
		Analog zur asymmetrischen Verschlüsselung,
		kann mit dem privaten Schlüssel etwas signiert werden.
		Mit dem öffentlichen Schlüssel lässt sich die Signatur überprüfen.
		Da der private Schlüssel nur dem Besitzer des Zertifikates bekannt ist,
		können von ihm erstellte Proxy-Zertifikate auf ihre Gültigkeit überprüft werden.
		
		Auf die zeitlich beschränkten Proxy-Zertifikat gehen Zugriffsrechte des Ausstellenden Nutzers über,
		was als Credentials-Delegation bezeichnet wird.
		Mit Hilfe der Proxy-Zertifikate können Dienste und Jobs agieren
		als wären sie der Ausstellende Nutzer
		Anhand des öffentlichen Schlüssels des Nutzer kann die Authentizität 
		der Zertifikate und die Nutzerzuordnung erfolgen.

	\subsubsection{Prüfen von Grid-Zertifikaten}
		Mit dem Befehl: \\
		\texttt{openssl x509 -text -in /clusterwork/sda1/home/griduser9/.globus/usercert.pem}
		lässt sich das Zertifikat anzeigen und überprüfen.
		Zur gleichen Ausgabe führt der Befehl \texttt{grid-cert-info}.\\
		Der Zertifikatsname (DN) ist \texttt{O=org, OU=Instant-Grid, OU=TestGrid, CN=griduser9}.
		Der Ausgeber des Zertifikates ist \texttt{O=org, OU=Instant-Grid, OU=TestGrid, CN=Instant-Grid CA}
		und es ist gültig von 21. Mai 2012, 09:37:35 GMT bis 21. Mai 2013, 09:37:35 GMT.
		
	\subsubsection{Prüfen von Grid-Proxy-Zertifikaten}
		Mit dem Befehl:
		\texttt{openssl x509 -text -in /tmp/x509up\_1013}
		lässt sich das Proxy-zertifkat anzeigen und überprüfen.
		Zur einer ähnliche Anzeige führt der Befehl 
		\texttt{grid-proxy-info}.\\
		Der Zertifikatsname (DN) ist \texttt{O=org, OU=Instant-Grid, OU=TestGrid, CN=griduser9, CN=857000645}.
		Der Ausgeber des Zertifikates ist \texttt{O=org, OU=Instant-Grid, OU=TestGrid, CN=griduser9}
		und es ist gültig vom 31. Mai 2012, 11:05:06 GMT bis zum 31. Mai, 23:10:06 GMT.
		
	\subsubsection{Verändern von Grid-Proxy Zertifikaten}
		Mit dem Befehl
		\texttt{grid-proxy-init -valid 168:0}
		lässt sich das Zertifikat um eine Woche verlängern.
		Um das Proxy-Zertifikat zu löschen kann der Befehl
		\texttt{grid-proxy-destroy}
		verwendet werden.

\subsection{Grid-Umgebung}
	\subsubsection{Service Container der Grid-Umgebung}
		Im Container laufen die in Tabelle \ref{tab:containerServices} aufgeführten Dienste.
		\begin{table}[ht!]
			\begin{center}
			\begin{tabular}{l l} 
			\toprule
			\multicolumn{2}{ c }{Dienste}	\\
			\midrule
				AuthzCalloutTestService				&		AdminService				\\
				CASService		&                       SampleAuthzService		 \\ 
				ContainerRegistryEntryService		&     SecureCounterService		 \\
				ContainerRegistryService		&        SecurityTestService		  \\
				CounterService		&                    ShutdownService			  \\
				DefaultIndexService		&              SubscriptionManagerService\\
				DefaultIndexServiceEntry		&        TestAuthzService			  \\
				DefaultTriggerService		&           TestRPCService				\\
				DefaultTriggerServiceEntry		&        TestService				\\
				DelegationFactoryService		&        TestServiceRequest		 \\
				DelegationService		&                 TestServiceWrongWSDL		  \\
				DelegationTestService		&           TriggerFactoryService	  \\
				InMemoryServiceGroup		&              TriggerService				\\
				InMemoryServiceGroupEntry		&        TriggerServiceEntry		 \\
				InMemoryServiceGroupFactory		&     Version				\\
				IndexFactoryService		&              WidgetNotificationService \\
				IndexService		&                    WidgetService				\\
				IndexServiceEntry		&                 AuthenticationService	\\
				ManagedExecutableJobService		&     IndexService				\\
				ManagedJobFactoryService		&        IndexServiceEntry			 \\
				ManagedMultiJobService		&           IndexService				\\
				ManagementService		&                 IndexServiceEntry			\\
				NotificationConsumerFactoryService	  &PersistenceTestSubscriptionManager		\\
				NotificationConsumerService		&     ReliableFileTransferFactoryService		\\
				NotificationTestService		&           ReliableFileTransferService		\\
				RendezvousFactoryService		& 	\\
			\bottomrule
			\end{tabular}
			\caption{Liste aller laufenden Dienste im Service Container.}
			\label{tab:containerServices}
			\end{center}
		\end{table}

	\subsubsection{Verfügbarkeit von Rssourcen in der Grid-Umgebeung}
		In der Textdatei \texttt{/usr/bin/machines} stehen die Namen der verfügbaren Rechner.
		Sie wird alle 30 Sekunden aktualisiert.\\
		Mit dem Befehl \texttt{globus-job-run bombay /bin/hostname -f} lässt sich ein Job auf der 
		Ressource \glqq bombay\grqq{} starten. Der Befehl im obrigen Beispiel ist \texttt{/bin/hostname -f}.
		
	\subsubsection{Datenverfügbarkeit in der Grid-Umgebung}
		Mit \texttt{globus-url-copy} lassen sich Dateien ins und im Grid bewegen.
		Beispielsweise die Datei \glqq moo\grqq{} nach \glqq bombay\grqq{}:\\
		 \texttt{globus-url-copy -vb file:///\$PWD/moo gsiftp://bombay/\textasciitilde/}.

\subsection{Job Submission mit dem Globus Toolkit}
	% $>globusrun-ws submit -f job.xml
	\subsubsection{Unterschied zwischen Fork und PBS Job}
	\subsubsection{Dokumentieren von Grid-Jobs}
		% MDS für Monitoring und Discovery
		% $>wsrf-query

