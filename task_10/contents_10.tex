\section{Aufgabenblatt 10}

\subsection{Assignment 1}
	\subsubsection{Blocking and non-blocking operations}
		The lecture slides define a blocking operation in the following way: 
		\textit{'Blocking: The return of the operation on a process ensures that the local resources (buffers) can be reused.'}
		That means that a process cannot perform any other operation on the local buffers until the previous, blocking operation completly finished and the buffer is free to be reused. 
		
		In contradiction, \textit{'a non-blocking operation does not guarantee that the local resources can be reused after returning to the process.} That results into a behaviour that the process, that calls an operation cannot be sure if the buffer is reusable again because the operation instantly returns the control flow to the process.
			
	\subsubsection{Collective and point-to-point communication}
				

	
\subsection{Assignment 2}
	\subsubsection{Running the hello world example}
After running the example from helloworld.c the output looked like this:
\begin{verbatim}
Hello MPI from the server process!
Hello MPI!
 mesg from 1 of 8 on c008
Hello MPI!
 mesg from 2 of 8 on c007
Hello MPI!
 mesg from 3 of 8 on c006
Hello MPI!
 mesg from 4 of 8 on c005
Hello MPI!
 mesg from 5 of 8 on c004
Hello MPI!
 mesg from 6 of 8 on c002
Hello MPI!
 mesg from 7 of 8 on c001
\end{verbatim}
Here we can see, that 

